\documentclass[11pt,fleqn]{article}

\usepackage[left=34mm,right=34mm,top=45mm,bottom=60mm,centering]{geometry}
\usepackage[english]{babel}
\usepackage[utf8x]{inputenc}
\usepackage{amsmath,amssymb,amsthm}
\usepackage{graphicx}
\usepackage{mathtools}
\usepackage{bm}
\usepackage{framed}
\usepackage{hyperref}

\usepackage{tikz} 
\usetikzlibrary{arrows}

\usepackage{sectsty}
\sectionfont{\sffamily\normalsize}
\subsectionfont{\normalfont\itshape\sffamily\normalsize}
\chapterfont{\sffamily}

\newtheorem{definition}{Definition}

\newcommand{\tr}{\mathrm{tr}\,}
\newcommand{\Tr}{\mathrm{Tr}\,}
\renewcommand{\vec}[1]{\mathbf{#1}}
\renewcommand{\overline}[1]{\bar{#1}}
\renewcommand{\Re}[0]{\operatorname{Re}}
\renewcommand{\Im}[0]{\operatorname{Im}}
\renewcommand{\d}{\mathrm{d}}

\newcommand{\Rmu}{R}
\newcommand{\R}{\tilde{R}}


\setlength\parindent{0pt}
\setlength{\parskip}{4mm plus2mm minus2mm}


\usepackage{tocloft}
\renewcommand{\cfttoctitlefont}{\sffamily\bfseries\large}
%\renewcommand{\cftchapfont}{\scshape}

\renewcommand{\cftsecfont}{\normalsize\sffamily}
%\renewcommand{\cftsecleader}{\hfill}
\renewcommand{\cftsecleader}{\cftdotfill{\cftdotsep}}
\renewcommand{\cftsecpagefont}{\cftsecfont}

\renewcommand{\cftsubsecfont}{\small\sffamily}
\renewcommand{\cftsubsecleader}{\hfill}
\renewcommand{\cftsubsecleader}{\cftdotfill{\cftdotsep}}
\renewcommand{\cftsubsecpagefont}{\cftsubsecfont}



\begin{document}

\vspace*{20mm}

{
\sffamily
\huge
\textbf{Parameters of the \texttt{openQ*D} main programs}
\\
\rule{\textwidth}{1pt}
\\[2mm]
\large
adapted by: Agostino Patella, RC$^*$ collaboration
\hfill
March 2018
\\[10mm]
\normalsize
heavily based on: M. L\"uscher, \textit{Parameters of the openQCD main programs}, June 2016, \texttt{doc/openQCD-1.6/rhmc.pdf}
}


%\vspace{30mm}

%\FloatBarrier


\pagebreak

\tableofcontents

\pagebreak



\section{Introduction}

The \texttt{openQ*D} main programs have many adjustable parameters. Most
parameter values are passed to the programs at run time through a human-readable
parameter file. The only parameters that must be specified at compilation time
are the lattice sizes and the MPI process grid (see \texttt{include/global.h}
and \texttt{main/README.global}). Each program also has a few command-line
options.

Some of the parameters are of a fairly trivial kind and are omitted in this note
(see, instead, the \texttt{README} files that come with each program). Details
about the simulation algorithm and the exact normalization conventions employed
can be found in the documentation files in the \texttt{doc} directory.

Only the parameters used by the programs that generate configurations
(\texttt{iso1}, \texttt{qcd1}, \texttt{ym1}, \texttt{mxw1}) are discussed in
these notes. For explanations of the parameters used by the other programs
(\texttt{ms*}), the reader is referred to the corresponding \texttt{README}
files.



\section{Preliminaries}

The input parameter files are divided into sections such as
%
\begin{verbatim}
  [Solver 3]
  solver CGNE
  nmx 256
  res 1.0e-10
\end{verbatim}
%
In this case, the section describes the solver number 3 for the Dirac equation.
The solver program is \texttt{CGNE}, the conjugate-gradient (CG) algorithm for
the normal Dirac equation, while \texttt{nmx} specifies the maximal number of CG
iterations that may be performed and \texttt{res} the desired relative residue
of the calculated solutions.

Sections start with a title in square brackets \texttt{[...]} and the lines
within a section can be ordered arbitrarily. The section quoted above, for
example, could equivalently be written as
%
\begin{verbatim}
  [Solver 3]
  res 1.0e-10
  solver CGNE   # Should try a better solver
  nmx 256
\end{verbatim}
%
Any text appearing to the right of the number sign \texttt{\#} is considered to
be a comment. A section is delimited by its headline (the text in square
brackets) and the headline of the next section. Blank lines are ignored and
sections can appear in any order in the input file.

Once the program has read the parameter file, the data are entered into a
parameter database and can, from there, be retrieved by the subprograms that
depend on some of the parameter values. The database is administered by a set
of modules in the directory \texttt{modules/flags}. In case the exact meaning of
a particular parameter is unclear, it may be helpful to read the explanations on
the top of these program files.


\section{Actions}
\label{sec:actions}


The lattice theory is defined by the lattice sizes, the field content, the total
action and the boundary conditions. A description of the latter and, in the case
of the simulation programs, of all parts of the action must be included in the
parameter file.

\subsection{Field content}
\label{subsec:actions:fields}

The \texttt{openQ*D} program can simulate the SU(3)$\times$U(1), SU(3) or U(1)
gauge group, with or without dynamical fermions. The relevant main programs are
listed below with their respective field content.
%
\begin{center}
\begin{tabular}{r|cc}
  Program & Gauge group & Fermions \\
  \hline
  \hline
  \texttt{iso1} & SU(3)$\times$U(1) & yes \\
  \texttt{qcd1} & SU(3) & yes \\
  \texttt{ym1} & SU(3) & no \\
  \texttt{mxw1} & U(1) & no
\end{tabular}
\end{center}
% In some of the main programs the gauge group can be chosen in the input file,
% while for others the gauge group is fixed.
% %
% \begin{itemize}
%   \item \texttt{iso1} -- The gauge group is fixed to be SU($3$)$\times$U($1$).
% 
%   \item \texttt{qcd1}, \texttt{ym1}, \texttt{ms3} -- The gauge group is fixed to
%   be SU($3$).
% 
%   \item \texttt{mxw1}, \texttt{ms5} -- The gauge group is fixed to be U($1$).
% 
%   \item \texttt{ms1} -- The gauge group can be chosen in a section of the form
% %
% \begin{verbatim}
%   [Field parameters]
%   gauge SU(3)xU(1)         # Gauge group: SU(3), U(1), SU(3)xU(1)
% \end{verbatim}
% %
% 
%   \item \texttt{ms2}, \texttt{ms4}, \texttt{ms6} -- The gauge group can be
%   chosen in a section of the form
% %
% \begin{verbatim}
%   [Gauge group]
%   gauge SU(3)xU(1)         # Gauge group: SU(3), U(1), SU(3)xU(1)
% \end{verbatim}
% %
% 
% \end{itemize}


\subsection{Boundary conditions}
\label{subsec:actions:bc}


The boundary conditions for the gauge fields are defined in a dedicated section.
For instance
%
\begin{verbatim}
  [Boundary conditions]
  type    3               # (equivalently) type periodic
  cstar   1
\end{verbatim}
%
selects periodic boundary conditions in time and C$^\star$ boundary conditions
in one space direction (and periodic in the others).

The value of the parameter \texttt{cstar} must be a number from 0 to 3, equal to
the number of spatial directions with C$^\star$ boundary conditions
\cite{cstar}. The gauge field is periodic in the remaining spatial direction.

Boundary conditions in time \cite{gauge_action} are chosen by setting the value
of the parameter \texttt{type}. If a SF boundary exists, then the phases
determining the value of the fields at the boundary need to be provided as well.
In the case of open boundary conditions and no C$^\star$ boundary conditions,
the section should be
%
\begin{verbatim}
  [Boundary conditions]
  type    0               # (equivalently) type open
  cstar   0
\end{verbatim}
%
If SF boundary conditions are chosen, the parameter section
%
\begin{verbatim}
  [Boundary conditions]
  type    1               # (equivalently) type SF
  cstar   0
  su3phi  1.57 0.0        # relevant only if the SU(3) field is used
  su3phi’ 2.75 -1.89      # relevant only if the SU(3) field is used
  u1phi   1.2             # relevant only if the U(1) field is used
  u1phi'  -0.3            # relevant only if the U(1) field is used
\end{verbatim}
%
must include:
\begin{itemize}
  %
  \item the angles $\phi_\text{SU(3),1}$, $\phi_\text{SU(3),1}$ defining the
  boundary values of the SU(3) gauge field at time $0$ as the two values of the
  parameter \texttt{su3phi}; this line can be skipped if the SU(3) field is not
  used;
  %
  \item the angles $\phi'_\text{SU(3),1}$, $\phi'_\text{SU(3),1}$ defining the
  boundary values of the SU(3) gauge field at time $T$ as the two values of the
  parameter \texttt{su3phi'}; this line can be skipped if the SU(3) field is not
  used;
  %
  \item the angle $\phi_\text{U(1)}$ defining the boundary values of the U(1)
  gauge field at time $0$ as value of the parameter \texttt{u1phi}; this line
  can be skipped if the U(1) field is not used;
  %
  \item the angle $\phi'_\text{U(1)}$ defining the boundary values of the U(1)
  gauge field at time $T$ as value of the parameter \texttt{u1phi'}; this line
  can be skipped if the U(1) field is not used.
  %
\end{itemize}
%
If open-SF boundary conditions are chosen, the boundary values of the gauge
fields at time $T$ must be specified as before. The
parameter section then looks like
%
\begin{verbatim}
  [Boundary conditions]
  type    2               # (equivalently) type open-SF
  cstar   0
  su3phi’ 2.75 -1.89      # relevant only if the SU(3) field is used
  u1phi'  -0.3            # relevant only if the U(1) field is used
\end{verbatim}
%


\subsection{Gauge action parameters}
\label{subsec:actions:gaugeparms}


If the SU(3) field is used, the parameters of the SU(3) gauge action have to be
defined in the parameter section
%
\begin{verbatim}
  [SU(3) action]
  beta    6.0            # Inverse SU(3) gauge coupling
  c0      1.6667         # Coefficient of the plaquette term
                         # in the SU(3) gauge action
  cG      1.10           # SU(3) gauge action improvement coefficient
                         # at time 0
  cG'     1.05           # SU(3) gauge action improvement coefficient
                         # at time NPROC0*L0
  SFtype  0              # Type of SF boundary (this affects the
                         # LW gauge action):
                         #   0|orbifold|openQCD-1.4
                         #   1|AFW-typeB|openQCD-1.2
\end{verbatim}
%
where \texttt{beta} is the parameter $\beta = 6/g_0^2$, \texttt{c0} is the $c_0$
parameter appearing in the L\"uscher-Weisz action, \texttt{cG} and \texttt{cG'}
are the boundary-improvement coefficients $c_\text{G}$ and $c'_\text{G}$
respectively, \texttt{SFtype} determines the particular form of the
double-plaquettes at the SF boundary if the L\"uscher-Weisz action is used. More
details on these parameters can be found in \cite{gauge_action}.

Which paramters are requested depend on the choice of boundary conditions in
time: with periodic boundary conditions
%
\begin{verbatim}
  [SU(3) action]
  beta    6.0
  c0      1.6667
\end{verbatim}
%
with open boundary conditions
%
\begin{verbatim}
  [SU(3) action]
  beta    6.0
  c0      1.6667
  cG      1.10    
\end{verbatim}
%
with SF boundary conditions and L\"uscher-Weisz action
%
\begin{verbatim}
  [SU(3) action]
  beta    6.0
  c0      1.6667
  cG      1.10
  SFtype  orbifold
\end{verbatim}
%
with open-SF boundary conditions and L\"uscher-Weisz action
%
\begin{verbatim}
  [SU(3) action]
  beta    6.0
  c0      1.6667
  cG      1.10
  cG'     1.05
  SFtype  orbifold
\end{verbatim}


If the U(1) field is used, the parameters of the U(1) gauge action have to be
defined in the parameter section
%
\begin{verbatim}
  [U(1) action]
  type    0              # Type of U(1) actions:
                         #   0|compact
                         #   1|non-compact
  alpha   0.05           # Fine-structure constant
  invqel  6.0            # Inverse of the elementary charge used in
                         # in the simulation. Quark electric charges
                         # must be even integer multiples of the
                         # elementary charge.
  c0      1.6667         # Coefficient of the plaquette term
                         # in the compact U(1) gauge action
  cG      1.10           # U(1) gauge action improvement coefficient
                         # at time 0
  cG'     1.05           # U(1) gauge action improvement coefficient
                         # at time NPROC0*L0
  SFtype  0              # Type of SF boundary (this affects the
                         # LW gauge action):
                         #   0|orbifold|openQCD-1.4
                         #   1|AFW-typeB|openQCD-1.2
\end{verbatim}
%
In the current version of the code, only the compact U(1) action is available
(parameter \texttt{type}). The value of \texttt{alpha} determines the
fine-structure constant $\alpha = e_0^2/(4\pi)$, the value of \texttt{invqel}
determines the parameter $q_\text{el}^{-1}$ which determines the normalization
of the U(1) gauge action, and consequently the quantum of electric charge.
\texttt{c0} is the $c_0$ parameter appearing in the L\"uscher-Weisz action,
\texttt{cG} and \texttt{cG'} are the boundary-improvement coefficients
$c_\text{G}$ and $c'_\text{G}$ respectively, \texttt{SFtype} determines the
particular form of the double-plaquettes at the SF boundary if the
L\"uscher-Weisz action is used. More details on these parameters can be found in
\cite{gauge_action}. As for the SU(3) gauge action, some parameters can be
omitted depending on the boundary conditions in time.



\subsection{Quark parameters}
\label{subsec:actions:quarkparms}


In the terminology of the \texttt{openQ*D} code, a \textit{quark flavour} is
identified by all adjustable parameters that define the Dirac operator. For
instance, in a simulation in the isospin symmetric limit, the up and down quark
count as a single \textit{quark flavour}. The number of quark flavours in this
sense is defined in the parameter section
%
\begin{verbatim}
  [Quark action]
  nfl          3
\end{verbatim}
%
Each quark flavour is identified by an integer in the range
$0,\dots,\texttt{nfl}-1$. The parameters defining the flavour 2 are defined in
the parameter section
%
\begin{verbatim}
  [Flavour 2]
\end{verbatim}
%
In case of periodic boundary conditions in time and space for the gauge field,
the quarks satisfy anti-periodic boundary conditions in time and phase-periodic
boundary conditions in space. The following parameter section has to be included
in case of a QCD simulation
%
\begin{verbatim}
  [Flavour 2]
  kappa        0.1300
  su3csw       1.234
  theta        0.5 1.0 -0.5
\end{verbatim}
%
where \texttt{kappa} is the hopping parameter, \texttt{su3csw} is the SW
coefficient $c_\text{sw}^\text{SU(3)}$, and the three values in \texttt{theta}
define the phase-periodic boundary conditions in the three special directions.
More details about these parameters are given in \cite{dirac}. If C$^\star$
boundary conditions are chosen in any of the spatial directions, the
\texttt{theta} parameters should be omitted. For a QCD+QED simulation with
(anti-)periodic boundary conditions in time and C$^\star$ boundary conditions in
space, the following parameter section has to be included
%
\begin{verbatim}
  [Flavour 2]
  qhat         -2
  kappa        0.1300
  su3csw       1.234
  u1csw        0.89
\end{verbatim}
%
where \texttt{qhat} is the electric charge in units of $q_\text{el}$ (defined in
the gauge action parameters) and \texttt{u1csw} is the SW coefficient
$c_\text{sw}^\text{U(1)}$.

If different boundary conditions in time are chosen, one needs to include also
the parameters \texttt{cF} and/or \texttt{cF'} defining the boundary improvement
coefficients $c_\text{F}$ and/or $c'_\text{F}$ respectively. For intance, a QCD
simulation with SF or open boundary conditions in time and C$^\star$ boundary
conditions in space should include
%
\begin{verbatim}
  [Flavour 2]
  kappa        0.1300
  su3csw       1.234
  cF           0.95
\end{verbatim}
%
Instead a QCD+QED simulation with open-SF boundary conditions in time and
C$^\star$ boundary conditions in space should include
%
\begin{verbatim}
  [Flavour 2]
  qhat         -2
  kappa        0.1300
  su3csw       1.234
  u1csw        0.89
  cF           0.95
  cF'          0.90
\end{verbatim}




\subsection{Gauge actions}
\label{subsec:actions:gauge}


Currently the supported gauge actions include the Wilson plaquette action, the
tree-level improved Symanzik action and, more generally, any linear combination
of these \cite{gauge_action}, both for the SU(3) and U(1) gauge field. Since the
parameters of the gauge actions are already specified in the \texttt{Boundary
conditions}, \texttt{SU(3) action} and \texttt{U(1) action} sections, it is
enough to include the lines
%
\begin{verbatim}
  [Action 0]
  action ACG_SU3
\end{verbatim}
%
in the parameter file to instruct the program that the total action of the
theory includes the SU(3) gauge action. This section is the first example of an
action section. It has a single line containing the parameter \texttt{action}
with value \texttt{ACG\_SU3}, which specifies that the action number 0 is the
SU(3) gauge action. Similarly it is enough to include the lines
%
\begin{verbatim}
  [Action 1]
  action ACG_U1
\end{verbatim}
%
in the parameter file to instruct the program that the total action of the
theory includes the U(1) gauge action as action number 1.


\subsection{Twisted-mass pseudo-fermion actions}
\label{subsec:actions:tm}

Pseudo-fermion actions
%
\begin{gather}
   S_\text{pf} = ( \phi, (D^\dag D + \mu_0^2)^{-1} \phi )
   \label{eq:action:tw1}
\end{gather}
%
with a single twisted-mass parameter $\mu_0$ are described by an action section
%
\begin{verbatim}
  [Action 2]
  action       ACF_TM1
  ipf          0
  ifl          2
  imu          1
  isp          0
\end{verbatim}
%
The number in the headline is an index that serves as a tag for the different
actions. It can be chosen arbitrarily in the range $0,1,\dots,31$, but must be
unambiguous, i.e. there may be at most one action section in the parameter file
with a given index. The action \eqref{eq:action:tw1} depends on the
pseudo-fermion field $\phi$, the quark flavour index \texttt{ifl} and the
twisted mass $\mu_0$. Pseudo-fermion fields are distinguished by an index
$\texttt{ipf}=0,1,\dots,\texttt{npf}-1$, where \texttt{npf} is the total number
of these fields. The quark flavour index \texttt{ifl} is a number in the range
$0,1,\dots,\texttt{nfl}-1$, and determines that the Dirac-operator parameters to
be used for this action are the ones defined in the corresponding
\texttt{Flavour} section. Twisted masses are specified by giving their index
\texttt{imu} in an array of values defined elsewhere in the parameter section
\texttt{HMC parameters} (see subsect. \ref{subsec:md:hmc}). The last parameter in
the section, \texttt{isp}, is the index of the solver for the Dirac equation,
which is to be used by the program that computes the action
\eqref{eq:action:tw1} (see sect. \ref{sec:solvers} for the list of the available
solvers).

Hasenbusch pseudo-fermion actions
%
\begin{gather}
   S_\text{pf} = ( \phi, (D^\dag D + \mu_1^2) (D^\dag D + \mu_0^2)^{-1} \phi )
   \label{eq:action:tw2}
\end{gather}
%
with two twisted-mass parameters $\mu_0$ and $\mu_1$ are described by an action
section
%
\begin{verbatim}
  [Action 3]
  action ACF_TM2
  ipf 1
  ifl 0
  imu 0 1
  isp 1 0
\end{verbatim}
%
As explained in the notes \cite{openQCD:forces}, these actions arise when the
light-quark determinant is split into several factors. The two masses correspond
to the two entries on the lines with tag \texttt{imu} and \texttt{isp}. Note
that one needs to specify two solvers, the first for the Dirac equation with
twisted mass $\mu_0$ and the second for the equation with twisted mass $\mu_1$
(which must be solved when $\phi$ is generated at the beginning of the
molecular-dynamics trajectories).




\subsection{Even-odd preconditioned pseudo-fermion actions}

When even-odd preconditioning is used, the pseudo-fermion action
\eqref{eq:action:tw1} gets replaced by
%
\begin{gather}
   S_\text{pf} = ( \phi, (\hat{D}^\dag \hat{D} + \mu_0^2)^{-1} \phi ) - 2\ln \det D_\text{oo}
   \label{eq:action:eotw1}
   \ ,
\end{gather}
%
where $\hat{D}$ is the even-odd preconditioned Dirac operator and $D_\text{oo}$
the odd-odd part of the Dirac operator \cite{openQCD:dirac}. The factor $\det
D_\text{oo}$ is referred to as the \text{small determinant} and it is understood
that the pseudo-fermion field $\phi$ vanishes on the odd sites of the lattice.

Apart from the action line, the parameter section
%
\begin{verbatim}
  [Action 4]
  action ACF_TM1_EO_SDET
  ipf 2
  ifl 0
  imu 0
  isp 3
\end{verbatim}
%
describing the action \eqref{eq:action:eotw1} look the same as the ones
describing the actions without even-odd preconditioning. The same comment
applies to the section
%
\begin{verbatim}
  [Action 5]
  action ACF_TM2_EO
  ipf 3
  ifl 0
  imu 0 1
  isp 1 0
\end{verbatim}
%
that describe the even-odd preconditioned version 
%
\begin{gather}
   S_\text{pf} = ( \phi, (\hat{D}^\dag \hat{D} + \mu_1^2) (\hat{D}^\dag \hat{D} + \mu_0^2)^{-1} \phi )
   \label{eq:action:eotw2}
\end{gather}
%
of the Hasenbusch pseudo-fermion action (there is no small-determinant
contribution in this case).


\subsection{Rational function actions}

The RHMC algorithm is based on the rational approximation of the operator
$(\hat{D}^\dag \hat{D} + \hat{\mu}^2)^\alpha$ (see ref.~\cite{rhmc} for detailed
explanations). If $\alpha=-1/2$, then the code calculates the Zolotarev rational
functions internally. Such rational functions are defined by a section
%
\begin{verbatim}
  [Rational 0]
  power   -1 2
  degree 10
  range 0.025 6.02
  mu 0.01
\end{verbatim}
%
that specifies the degree (number of poles) of the function, the approximation
range on the axis of eigenvalues of $(\hat{D}^\dag \hat{D})^{1/2}$, and the
twisted mass $\hat{\mu}$. The range should be sufficiently large to include,
with probability practically equal to 1, the whole spectrum of the operator.
Rational approximations with $\alpha = p/q$ where $p$ and $q$ are integers are
defined by a section
%
\begin{verbatim}
  [Rational 1]
  power            -1 4
  degree           3
  range            2.74000000e-02 6.29000000e+00
  mu               1.00000000e-02
  delta            1.6178479482115836e-02
  A                3.27355183713657515998e-01
  nu[0]            4.00031119921775957238e+00
  mu[0]            2.35403583930007442859e+00
  nu[1]            5.44609163341205237963e-01
  mu[1]            3.37080943797074983337e-01
  nu[2]            7.85762611113028086596e-02
  mu[2]            4.69405290534691196913e-02
  x[0]             7.50760000000000055770e-04
  x[1]             1.88538259325339335910e-03
  x[2]             1.06896142321669003483e-02
  x[3]             7.11451548372154934929e-02
  x[4]             4.72348110596534420669e-01
  x[5]             3.11953320147559853837e+00
  x[6]             1.69536291736011719422e+01
  x[7]             3.95641000000000033765e+01
\end{verbatim}
%
The two values of \texttt{power} are two integers $p$ and $q$ such that $\alpha
= p/q$, the two values of \texttt{range} are the extrema of the approximation
range on the axis of eigenvalues of $(\hat{D}^\dag \hat{D})^{1/2}$, the value of
\texttt{mu} is the twisted mass $\hat{\mu}$, the value of \texttt{degree} is the
number of poles. All other values specify the rational function. This section
should not be filled by hand, instead it should be produced with the
\texttt{MinMax} code (in the \texttt{minmax} directory of the \texttt{openQ*D}
code) which implements the minmax approximation algorithm in multiple precision.
The general rational functions accepted by the \texttt{openQ*D} code are
described in \cite{rhmc}. A full documentation of the \texttt{MinMax} program
can be found in its \texttt{README} file.

The poles and zeros of the rational functions are ordered such that the larger
ones come first. For reasons of stability and performance, the rational function
should be split into a product of factors, each including a range of poles and
zeros \cite{rhmc}. The pseudo-fermion action associated with such a factor is
described by a section
%
\begin{verbatim}
  [Action 6]
  action ACF_RAT
  ipf 4
  im0 1
  irat 0 7 9
  isp 1
\end{verbatim}
%
The parameters \texttt{ipf}, \texttt{ifl} and \texttt{isp} in this section have
the same meaning as in the case of the twisted-mass pseudo-fermion actions,
while \texttt{irat} specifies the index of the rational function and the range
of poles included in the factor. Poles are counted from zero and pole ranges are
inclusive, i.e. \texttt{irat 0 7 9} selects the poles number 7, 8 and 9 from the
rational function with index 0.

Since even-odd preconditioning is used for the RHMC, the associated quark
determinants contain the small determinant $\det D_\text{oo}$. It is convenient
to combine this factor with the rational factor that contains the largest poles.
The corresponding section is
%
\begin{verbatim}
  [Action 7]
  action ACF_RAT_SDET
  ipf 5
  im0 1
  irat 0 0 6
  isp 4
\end{verbatim}
%
For a given rational function and flavour index \texttt{ifl}, the action
sections must be such that all poles and the small determinant are included
once. In the example discussed here, this is achieved with two factors, but one
is free to split the rational functions in more factors. Note that for each
factor a different pseudo-fermion field must be used.




\section{Molecular-dynamics parameters}
\label{sec:md}

The simulation algorithm implemented in the \texttt{openQ*D} package involves a
numerical integration of the molecular-dynamics equations that derive from the
chosen action. Apart from the action, this requires the integration scheme (the
\textit{integrator}) to be specified as well as the solvers used for the
computation of the various pseudo-fermion forces.

\subsection{HMC parameters}
\label{subsec:md:hmc}

Some basic parameters of the algorithm are collected in the section
%
\begin{verbatim}
  [HMC parameters]
  actions 0 1 2
  npf 6
  mu 0.015 0.2 1.0
  nlv 2
  tau 1.0
  facc 1
\end{verbatim}
%
The numbers on the action line are the indices of the actions that are to be
included in the molecular-dynamics Hamilton function. If some of these actions
depend on twisted mass parameters, their values must be listed on the line with
tag \texttt{mu}. As already mentioned, the some of the action sections refer to
these values by quoting their position \texttt{imu} (counting from 0) in the
array. The parameter \texttt{facc} specifies whether the Fourier acceleration
should be used for the U(1) field. The parameter \texttt{npf} specifies the
total number of pseudo-fermion fields that must be allocated. The last two
parameters, \texttt{nlv} and \texttt{tau}, define the number of integrator
levels and the molecular-dynamics trajectory length (see ref.
\cite{openQCD:forces} for the normalization of the latter; the integration
scheme is discussed in subsect. \ref{subsec:md:integrator}).



\subsection{Forces}
\label{subsec:md:forces}

The forces that derive from the actions
\texttt{ACG\_SU3},...,\texttt{ACF\_RAT\_SDET} discussed in sect.
\ref{sec:actions} are distinguished by the symbols
\texttt{FRG\_SU3},...,\texttt{FRF\_RAT\_SDET}. The corresponding sections are
%
\begin{verbatim}
  [Force 0]
  force FRG_SU3
\end{verbatim}

\begin{verbatim}
  [Force 1]
  force FRG_U1
\end{verbatim}

\begin{verbatim}
  [Force 2]
  force FRF_TM1
  isp 6
  ncr 4
\end{verbatim}

\begin{verbatim}
  [Force 3]
  force FRF_TM2
  isp 7
  ncr 0
\end{verbatim}

\begin{verbatim}
  [Force 4]
  force FRF_TM1_EO_SDET
  isp 6
  ncr 4
\end{verbatim}

\begin{verbatim}
  [Force 5]
  force FRF_TM2_EO
  isp 7
  ncr 0
\end{verbatim}

\begin{verbatim}
  [Force 6]
  force FRF_RAT
  isp 8
\end{verbatim}

\begin{verbatim}
  [Force 7]
  force FRF_RAT_SDET
  isp 9
\end{verbatim}
%
In each case, the index in the headline must match the index of the associated
action. The force tags (such as \texttt{FRF\_TM2\_EO}) could be inferred from
the corresponding action sections, but are required in order to improve the
readability of the input parameter files.

Most parameters of the forces are inherited from those of the associated
actions. The solver parameter sets selected by the indices \texttt{isp} may
however be different. In the case of the forces \texttt{FRF\_TM2} and
\texttt{FRF\_TM2\_EO}, the specified solver is used for the solution of the
Dirac equation with twisted mass $\mu_0$. No solver is needed here for the
second twisted mass $\mu_1$.

The parameter \texttt{ncr} controls the chronological propagation of the
solutions of the Dirac equation along the molecular-dynamics trajectories. If
\texttt{ncr} is set to a positive value, the simulation program attempts to
reduce the number of solver iterations required for the computation of the
specified force by extrapolating the previous \texttt{ncr} solutions in
molecular-dynamics time. The feature is switched off if \texttt{ncr} is set to
zero.





\subsection{Integration scheme}
\label{subsec:md:integrator}

The numerical integration of the molecular-dynamics equations makes use of a
hierarchical integrator that can be specified on the input parameter file.
Currently three elementary integration schemes are supported, the leapfrog, the
2nd order Omelyan–Mryglod–Folk (OMF) and a 4th order OMF integrator. They are
distinguished by the symbols \texttt{LPFR}, \texttt{OMF2} and \texttt{OMF4}.

Hierarchical integrators have several levels with increasing integration step
sizes. They are described by the total integration time tau, the number nlv of
levels, the elementary integrators used at each level and the forces integrated
there. The parameters tau and nlv are part of the HMC parameter set (see
subsect. \ref{subsec:md:hmc}). Each level is then described by a section like
%
\begin{verbatim}
  [Level 2]
  integrator LPFR
  nstep 12
  forces 2 4 5
\end{verbatim}
%
In this example, the simulation program is instructed to use the leapfrog
integrator at the third level (levels are counted from 0). The elementary
leapfrog integration step is to be applied 12 times and the forces integrated at
this level are the ones with index 2, 4 and 5. If the level is the top level,
the integration step size for these forces is thus \texttt{tau}/12.

An example of a complete integrator description is provided by the following
three sections:
%
\begin{verbatim}
  [Level 0]
  integrator OMF4
  nstep 1
  forces 0
\end{verbatim}

\begin{verbatim}
  [Level 1]
  integrator OMF4
  nstep 1
  forces 1 2 3
\end{verbatim}

\begin{verbatim}
  [Level 2]
  integrator OMF2
  lambda 0.16667
  nstep 6
  forces 4
\end{verbatim}
%
There are three levels in this case, with average step sizes equal to
\texttt{tau}/300, \texttt{tau}/60 and \texttt{tau}/12 at level 0, 1 and 2,
respectively (note that the \texttt{OMF2} and \texttt{OMF4} integrators update
the gauge field 2 and 5 times per application). The OMF2 integrator depends on a
parameter \texttt{lambda}, whose value must be given at each level where this
integrator is used. See \texttt{modules/update/README.mdint} for further
explanations.



\section{Solver parameters}
\label{sec:solvers}


All solvers referred to in the action and force sections must be described in
the input parameter file. The solvers are labeled by an index \texttt{isp} in
the range 0,1,...,31 in much the same way as flavours, actions and forces. There
is no one-to-one correspondence between solver programs and solver sections in
the parameter file. One may, for example, have two sections
%
\begin{verbatim}
  [Solver 3]
  solver CGNE
  nmx 256
  res 1.0e-10
\end{verbatim}

\begin{verbatim}
  [Solver 4]
  solver CGNE
  nmx 256
  res 1.0e-8
\end{verbatim}
%
where the only difference is the value of the desired residue \texttt{res}. By
setting $\texttt{isp}=3$ or $\texttt{isp}=4$ in an action or force section, the
numerical accuracy of the action and force computations can then be individually
controlled.


\subsection{Conjugate gradient solvers}
\label{subsec:solvers:cg}

There are two conjugate gradient solvers, the ordinary CG algorithm for the
normal Dirac equations,
%
\begin{gather}
   (D^\dag D + \mu^2) \psi = \eta \quad \text{and} \quad (\hat{D}^\dag \hat{D} + \mu^2) \psi = \eta \ ,
\end{gather}
%
and the mult-shift CG algorithm for the even-odd preconditioned simultaneous
equations \cite{openQCD:mscg}
%
\begin{gather}
   (\hat{D}^\dag \hat{D} + \mu_k^2) \psi_k = \eta \qquad k=0,1,\dots,n-1 \ .
\end{gather}
%
Examples of parameter sections describing solvers based on the former have
already appeared in this note. The parameter sections
%
\begin{verbatim}
  [Solver 3]
  solver MSCG
  nmx 256
  res 1.0e-10
\end{verbatim}
%
for the multi-shift solver look practically the same. Note that this solver can
only be used for rational function actions and forces.



\subsection{SAP preconditioned solvers}
\label{subsec:solvers:sap}

There are currently two solvers that make use of the Schwarz Alternating
Procedure (SAP) as preconditioner for the GCR algorithm. When these solvers are
used, the block size bs of the SAP block grid needs to be specified in a
separate section
%
\begin{verbatim}
  [SAP]
  bs 4 6 6 4
\end{verbatim}
%
The ordinary SAP preconditioned GCR algorithm is then described by a parameter
section
%
\begin{verbatim}
  [Solver 7]
  solver SAP_GCR
  nkv 16
  isolv 1
  nmr 4
  ncy 5
  nmx 24
  res 1.0e-8
\end{verbatim}
%
where \texttt{nkv} is the maximal number of Krylov vectors that may be generated
before the GCR algorithm is restarted, while \texttt{nmx} is the maximal total
number of generated Krylov vectors and res the desired relative residue of the
calculated solutions. All other parameters in the section describe the
particular SAP preconditioner to be used, namely \texttt{ncy} and \texttt{nmr}
specify the number of SAP cycles and block solver iterations to be applied and
the flag \texttt{isolv} indicates whether the block solver is even-odd
preconditioned (\texttt{isolv}=1) or not (\texttt{isolv}!=1).

The other solver that makes use of the SAP is the deflated SAP preconditioned
GCR solver. This solver is described by a section
%
\begin{verbatim}
  [Solver 8]
  solver DFL_SAP_GCR
  idfl 0
  nkv 16
  isolv 1
  nmr 4
  ncy 5
  nmx 24
  res 1.0e-8
\end{verbatim}
%
that coincides with the section for the ordinary SAP preconditioned solver
except for the solver symbol, and for the parameter \texttt{idfl} which
specifies which deflation subspace should be used (see
subsect.~\ref{subsec:solvers:dfl}).


\subsection{Deflation parameters}
\label{subsec:solvers:dfl}

When the deflated solver is used, the parameters related to the deflation
subspace must be specified in a few further sections.

In contrast to \texttt{openQCD}, the \texttt{openQ*D} package can use seveal
deflation subspaces simultaneously. This is useful in QCD+QED simulation, since
quarks with different electric charges must use different deflation subspaces.
The deflation subspaces are labeled by an index \texttt{idfl} in the range
0,1,...,31 in much the same way as flavours, actions, forces and solvers. In
practice this index is used only in the parameters sections of the
\texttt{DFL\_SAP\_GCR} solver (see subsect.~\ref{subsec:solvers:sap}). The
deflation parameters are distributed in four different sections. The of them,
\texttt{Deflation subspace}, \texttt{Deflation projection} and \texttt{Deflation
update scheme} are common to all deflation subspaces, while \texttt{Deflation
subspace generation} must appear as many times as the number of requested
deflation subspaces.

The block size \texttt{bs} of the deflation block grid and the number
\texttt{Ns} of deflation modes per block are set by including the section
%
\begin{verbatim}
  [Deflation subspace]
  bs 8 4 4 4
  Ns 20
\end{verbatim}

For the generation of the deflation subspace, approximate eigenvalues of the
Dirac operator are calculated. Each deflation subspace can use different
parameters for the Dirac operator and different values for the twisted mass
\texttt{mu}, which need to be specified in the section
%
\begin{verbatim}
  [Deflation subspace generation 2]
  qhat     0             # Dirac operator parameters start here
  kappa    0.128
  su3csw   1.234
  u1csw    0.000
  cF       0.95
  cF'      1.0
  theta    0.2 -0.4 0.5  # Dirac operator parameters end here
  mu       0.02
  ninv     5
  nmr      4
  ncy      5
\end{verbatim}
%
The other parameters set in this section include the number \texttt{ninv} of
inverse iteration steps to be applied and the numbers \texttt{ncy} and
\texttt{nmr} of SAP cycles and block solver iterations to be used when the SAP
preconditioner is invoked.

In the course of the inverse iteration, and in each iteration of the deflated
solver, a deflation projection is applied that requires the little Dirac
equation to be solved with low precision. The section
%
\begin{verbatim}
  [Deflation projection]
  nkv 24
  nmx 256
  res 1.0e-2
\end{verbatim}
%
sets the parameters of the GCR solver used for this task. As usual,
\texttt{nkv}, \texttt{nmx} and \texttt{res} are, respectively, the number of
Krylov vectors generated before the GCR algorithm is restarted, the maximal
total number of vectors that may be generated and the desired relative residue
of the calculated solution.

Along the molecular-dynamics trajectories, the deflation subspace looses its
efficiency and must therefore be refreshed from time to time. This feature is
controlled by the parameter section
%
\begin{verbatim}
  [Deflation update scheme]
  dtau 0.091
  nsm 1
\end{verbatim}
%
where \texttt{dtau} sets the interval in molecular-dynamics time after which the
subspace is updated and \texttt{nsm} the number of smoothing steps to be applied
in this process. The subspace updates can be turned off by setting
\texttt{nsm}=0. Clearly, the section is not needed in measurement programs,
where the deflated solver may be used for quark propagator calculations, for
example.




\section{Reweighting factors}
\label{sec:rw}

In general, the ensembles of field configurations generated in \texttt{openQ*D}
simulations must be reweighted. For example, the rational approximation requires
reweighting in order to correct for the approximation error. In the case of the
light quarks, reweighting is required if a twisted mass term was added as
infrared regulator. For a given ensemble of field configurations, the program
\texttt{ms1} computes stochastic estimates of the reweighting factors specified
in a parameter file.


\subsection{Twisted-mass reweighting factors for doublets}

In ref. \cite{Luscher:2008tw}, two kinds of twisted-mass regularizations of the
determinant $\det \{ D^\dag D \}$ were proposed which amount to replacing
%
\begin{gather}
   \det \{ D^\dag D \} \rightarrow \det \{ D^\dag D + \mu^2 \}
\end{gather}
and
\begin{gather}
   \det \{ D^\dag D \} \rightarrow \det \{ (D^\dag D + \mu^2)^2 (D^\dag D + 2\mu^2)^{-1} \}
\end{gather}
%
respectively. The twisted mass parameter $\mu > 0$ provides the desired infrared
regularization and is usually set to a value on the order of light quark mass.

The associated reweighting factors,
%
\begin{gather}
   W_1 = \det \det \{ D^\dag D  (D^\dag D + \mu^2)^{-1} \} \ , \\
   W_2 = \det \{ D^\dag D (D^\dag D + 2\mu^2) (D^\dag D + \mu^2)^{-2} \}
\end{gather}
%
can be estimated stochastically using Gaussian random quark fields. The input
files parameters for these reweigthing factors are identical to the
\texttt{openQCD} package and are described in sect. 6.1 of \cite{openQCD:parms}.



\subsection{Reweighting factors for RHMC}

Let $\R$ and $\Rmu$ be the optimal rational approximations of order $[n,n]$ for
$(\hat{D}^\dag \hat{D})^{-\alpha}$ and $(\hat{D}^\dag \hat{D} +
\hat{\mu}^2)^{-\alpha}$ respectively. As explained in \cite{rhmc}, two
reweighting factors are needed to correct for the twisted mass in the rational
approximation ($W_\text{rtm}$), and for the error due to the rational
approximation itself ($W_\text{rat}$).


The calculation of the reweighting factor
%
\begin{gather}
   W_\text{rtm} = \det [ \R^{-1} \Rmu ]
   \label{eq:Wtm}
\end{gather}
%
requires two rational functions, $\Rmu$ and $\R$, that need to be defined in two
parameter sections of the type
%
\begin{verbatim}
  [Rational 0]
  power        -1 2
  degree       12
  range        0.02 6.05
  mu           0.0
\end{verbatim}

\begin{verbatim}
  [Rational 1]
  power        -1 2
  degree       12
  range        0.02 6.05
  mu           0.02
\end{verbatim}
%
The parameter section for the reweighting factor $W_\text{rtm}$ looks like
%
\begin{verbatim}
  [Reweighting factor 2]
  rwfact       RWRTM
  ifl          1
  irp          1 0
  np           6 4 2
  isp          1 0 2
  nsrc         4
\end{verbatim}
%
The parameter \texttt{ifl} is used to specify which flavour needs to be used.
The line \texttt{irp} declares that the rational approximation 1 is used for
$\Rmu$, and the rational approximation 0 is used for $\R$. It is required that
the two functions have the same degree (12 in this case). The operator
$\Rmu^{-1} \R$ is also a rational function of $\hat{Q}^2=\hat{D}^\dag \hat{D}$,
which can broken up in factors as explained in \cite{rhmc}. The line \texttt{np}
specifies that the first factor contains the first group of 6 poles and zeroes,
the second factor contains the second group of 4 poles and zeroes, and finally
the third factor contains the third group of 2 poles and zeroes, according to
the formula
%
\begin{gather}
   W_\text{rtm} = \text{constant} \times \det \tilde{P}_{1,5}^{-1} \ \det \tilde{P}_{6,10}^{-1} \ \det \tilde{P}_{11,12}^{-1} \ .
\end{gather}
%
The calculation of each determinant requires the solution of the Dirac equation.
Solvers can be chosen independently for each factor by setting the values of
\texttt{isp}. Each determinant is evaluated stochastically, using \texttt{nsrc}
stochastic sources.


The parameters for the reweighting factor
\begin{gather}
   W_\text{rat} = \det [ (\hat{D}^\dag \hat{D})^{\alpha} \R ]
   \label{eq:Wrat}
\end{gather}
are specified in a section of the type
%
\begin{verbatim}
  [Reweighting factor 1]
  rwfact       RWRAT
  ifl          1
  irp          0
  np           6 4 2
  isp          1 0 2
  nsrc         2
\end{verbatim}
%
The parameter \texttt{ifl} is used to specify which flavour needs to be used.
The rational function for $\R$ is specified in \texttt{irp}. Notice that the
exponent $\alpha$ is inherited from the rational function. The reweighting
factor is estimated stochastically with a procedure described in detail in
\cite{rhmc}. \texttt{nsrc} stochastic sources are used for this determination.
Every time the operator $\R$ needs to be applied to a stochastic sources, the
rational function is broken up in factors pretty much in the same way as in the
discussion for $W_\text{rtm}$. The splitting is defined in the line \texttt{np}.
For each factor, the solvers to be used are chosen independently in the line
\texttt{isp}.




\begin{thebibliography}{9}

\bibitem{cstar}
  A. Patella,
  \textit{C$^\star$ boundary conditions}, code documentation,
  \texttt{doc/cstar.pdf}.

\bibitem{gauge_action}
  A. Patella,
  \textit{Gauge actions}, code documentation,
  \texttt{doc/gauge\_action.pdf}.

\bibitem{dirac} 
  A.~Patella,
  \textit{Dirac operator}, code documentation,
  \texttt{doc/dirac.pdf}.

\bibitem{openQCD:forces} 
  M.~Luscher,
  \textit{Molecular-dynamics quark forces}, code documentation,
  \texttt{doc/openQCD-1.6/forces.pdf}.

\bibitem{openQCD:dirac} 
  M.~Luscher,
  \textit{Implementation of the lattice Dirac operator}, code documentation,
  \texttt{doc/openQCD-1.6/dirac.pdf}.

\bibitem{rhmc} 
  A. Patella,
  \textit{RHMC algorithm in \texttt{openQ*D}}, code documentation,
  \texttt{doc/rhmc.pdf}.

\bibitem{openQCD:mscg} 
  M.~Luscher,
  \textit{Multi-shift conjugate gradient algorithm}, code documentation,
  \texttt{doc/openQCD-1.6/mscg.pdf}.

\bibitem{Luscher:2008tw} 
  M.~Luscher and F.~Palombi,
  \textit{Fluctuations and reweighting of the quark determinant on large lattices},
  PoS LATTICE {\bf 2008}, 049 (2008)
  [arXiv:0810.0946 [hep-lat]].

\bibitem{openQCD:parms} 
  M.~Luscher,
  \textit{Parameters of the openQCD main programs}, code documentation,
  \texttt{doc/openQCD-1.6/parms.pdf}.

\end{thebibliography}



\end{document}
